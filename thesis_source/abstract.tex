\begin{abstractzh}
在工業檢測的場合中,自動化影像辨識技術日漸成為一個重要的應用,
而現成的影像處理軟體與演算法通常有授權費過高,且準確度與分析速度並不符合產線的預算與需求,
對此問題,我們希望可以找到一個低成本的解決方案,同時滿足產線對於分析速度的需求。
本研究所實作出的產品檢驗流程可以粗略分為兩步驟,
第一步驟為樣板設定,此步驟會紀錄標準的產品特徵;
第二步驟為樣本檢驗,此步驟會將樣本與第一步驟所記錄下的樣本進行比對,並判斷此背光鍵盤是否有瑕疵或故障。
本論文主要討論自動化光學檢測系統及分析演算法的設計架構與分析過程中的演算法比較並加以改良。
並在最後將嘗試過的各種方法在產線的標準下進行比較。
\end{abstractzh}

\begin{abstracten}
The purpose of optical music recognition is to develop a computer program that is able to understand the musical score, which is invented for human beings to annotate melody. A score is usually stored as an image. Therefore, a recognition system must retrieve musical information from a set of pixels.
This dissertation deals with two major issues: preprocessing and recognition. Preprocessing aims at dividing the input image into several slices that can be processed independently and handling the defects in the printing step. The goal of preprocessing is to simplify the subsequent recognition stage. Afterward, recognition on a staff image is the core of this dissertation. The implementation is based on template matching and the support vector machine. For real score images, the present algorithm works well.
The design of the present algorithm brings a different perspective to optical music recognition. First, the preprocessing uses \emph{random sample consensus} (RANSAC) as a part of staff detection. Such randomness makes it meaningful to repeat the same operation; by comparing the results between different iterations, consensus-based correction provides possibility of finding symbols that other existing stable algorithms cannot find. Secondly, the algorithm is based on the \emph{divide and conquer} concept, which means the subtasks have little correlation, and hence the algorithm can be readily parallelized.
\end{abstracten}

% \keywords{Optical Music Recognition, Pattern Recognition, Music Technology}

\begin{comment}
\category{I2.10}{Computing Methodologies}{Artificial Intelligence --
Vision and Scene Understanding} \category{H5.3}{Information
Systems}{Information Interfaces and Presentation (HCI) -- Web-based
Interaction.}

\terms{Design, Human factors, Performance.}

\keywords{Region of interest, Visual attention model, Web-based
games, Benchmarks.}
\end{comment}
