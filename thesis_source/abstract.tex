\begin{abstractzh}
在工業檢測的場合中,自動化影像辨識(AOI)技術是一種常見的應用,
而現成的影像處理軟體與設備通常有授權費與設置成本過高,並且缺乏日後修改的彈性,且準確度與分析速度並不符合產線的預算與需求,
這份論文中,我們與羅技電子合作以OpenCV為基礎重新設計了檢測方法,利用多台網路攝影機取代工業級攝影機實作了一個低成本的解決方案。
本研究所實作出的產品檢驗流程可以用來檢測背光鍵盤的刻字以及燈光顏色是否符合出貨標準。\\
檢測流程可以粗略分為兩步驟,
第一步驟為標準設定,此步驟會輸入一張符合出貨標準的產品影像,紀錄特徵,例如刻字的形狀與LED燈色的標準,並在輸入的影像中標記SURF特徵點;
第二步驟為樣本檢驗,此步驟會使用SURF特徵點與FLANN演算法,將樣本與第一步驟所記錄下的標準影像進行對齊(Alignment)與瑕疵檢測(Defect Detection),並判斷此背光鍵盤樣本是否有瑕疵或故障。
詳細的流程將會在第三章進行介紹。\\
本論文將在第四章比較應用於此自動化光學檢測系統的特徵點演算法,並在最後針對各種產線上可能的的干擾進行模擬測試,並選出一個適合的演算法投入產線使用。\\
在第五章,我們將會對目前實作的成果進行總結並提出未來展望。

\end{abstractzh}

\begin{abstracten}
In the application of product inspection and quality control, automatic optical inspection (AOI) is an common application.
But with the ready-made AOI softwares and instruments, comes the higher deployment cost and less flexibility when the requirements are changed, Also, the accuracy and inspection time doesn't meet the need for the production site.
In this thesis, we cooperate with Logitech, redesigned a AOI solution based on OpenCV.
We utilized multiple web-cams to replace expensive industrial camera, constructed a lower cost solution.
The AOI system will be discussed in the thesis is designed to inspect the lettering quality \& color of LED back light of the keyboards from Logitech. To see whether the product meets the shipping standard or not.\\
The inspection process can roughly divide into two main parts, the standard setup phase and the inspection phase.
In the first phase, standard setup phase, we'll input an image (reference image) of good product that meets shipping standard, then record all the information required for the inspection phase, like the lettering shape and LED colors, and do the SURF feature point detection on the reference image.
In the second phase, inspection phase, we'll input an image (sample image) of inspection sample, apply SURF feature point detection on the image and align the sample image with reference image. Then we'll apply the defect detection methods to see whether this sample having any defect or not.
The detailed methods would be introduced in Ch.3.\\
In the Ch.4, we'll compare the different feature point detection methods, and design an experiment to simulate the possible interference, than choose a best feature point detector for our application.\\
In the Ch.5, we'll made conclusions and future works for the current system implementation. 

\end{abstracten}