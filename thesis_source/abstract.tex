\begin{abstractzh}
在工業檢測的場合中,自動化影像辨識(AOI)技術日漸成為一個重要的應用,
而現成的影像處理軟體通常有授權費過高,且準確度與分析速度並不符合產線的預算與需求,
對此問題,我們希望可以找到一個低成本的解決方案,同時滿足產線對於分析速度的需求。
本研究所實作出的產品檢驗流程可以粗略分為兩步驟,
第一步驟為樣板設定,此步驟會紀錄標準的產品特徵;
第二步驟為樣本檢驗,此步驟會將樣本與第一步驟所記錄下的樣本進行比對,並判斷此背光鍵盤是否有瑕疵或故障。
本論文主要討論自動化光學檢測系統及分析演算法的設計架構與分析過程中的演算法比較並加以改良。
並在最後將嘗試過的各種方法以產線實際運作的標準下進行比較。

\end{abstractzh}

\begin{abstracten}
In the industrial production site, Automatic-Optics-Inspection (AOI) techniques has become an importent application.
Since the existing image processing software is not cost effective due to expensive licence fee, 
and doesn't meet the requirment of on speed and accuracy. 
In this thesis, we proposed a low cost solution implemented with open source tool

\end{abstracten}