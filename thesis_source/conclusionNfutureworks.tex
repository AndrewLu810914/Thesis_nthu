\chapter{Conclusions \& Future Works}
\label{c:conclusion-and-future-works}

\section{Conclusions}
	In this thesis, we have discussed the performance bottleneck of the inspection methods and improved the inspection methods.
	By the experiment \& comparision, we found that both SIFT \& SURF feature point detector have very good robustness.
	Thought AKAZE have slightly benefit on time efficiency then SURF, the complexity of paraneter setting and robustness doesn't meet the production lines requirment.

 	The current implementation solved the performance issue of the previous solution, increased the accuracy and precision of the judgment by redesigning the inspection process.
	But this implementation doesn't consider the case of conventional (non-iluminated) keyboards, also doesn't utilized any parallel computing techniques.
	We will discuss about the future work in the following section.

\section{Future Works}
	\subsection{Ability To Inspect Non-illuminated Keyboards}
		For the key board lettering inspection method, since the current method cannot handle the shading that is too complex.

	\subsection{Multi-camera and controls optimization}
		The current version of implementation using the multi-camera in sequential order. Parallel processing techniques is not utilized into this project yet.
		And every camera we using is now using a manual adjusted parameter setup. Witch means the system performance is highly depends on the experience of the user of the system.


	\subsection{Parallel computing is not utilized in this project}
		In this thesis, we didn't fully utilize the computation power of the host PC, only use the sequential processing technique in the current implementation to ensure the stability of the system.
		The system may have an acceleration about the number of CPU cores in the ideal situation. 
		Witch is a great improvement for the production line application.
		And GPU is also a powerful tool witch we didn't introduce to this project since the budget issue.
		With good implementation, GPU may make the real time inspection possible.
	
	\subsection{Smarter Algorithm}
		In current implementation, we still need to setup up to ten parameters manually.
		And the current implementation is still sensetive to the imput image source and inspection sample properties.
		Thus a smarter methods considering more delicate condition is required in the future.