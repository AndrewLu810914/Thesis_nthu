\chapter{Comparison Between Other Methods}
\label{c:comparision}

\section{Different Feature Point Extraction \& Point Matching Methods}
	% test functions provides in opencv
	% https://docs.opencv.org/3.3.1/d5/d51/group__features2d__main.html
	In this section, we'll discuss how different feature point performs in keyboard defect detection scenario.
	Since the bottleneck of this project is the feature point detection method, we tried some common methods, and below is the observation \& time comparison between different approach.

	\subsection{Simple Blob Detector}
		\subsubsection{Introduction}
			The blob detection is an simple approach for feature point detection. We choose the centroid of each selected blob as the feature point. Matching feature points with nearest neighborhood. 
			Since we have a fixture that can guarantee the keyboard does not have shift or rotation more then a certain amount, we can assume that the feature point matched is the nearest one.
		\subsubsection{Performance}
			$$\textrm{Add time data}$$
		\subsubsection{Pros \& Cons}
			This method has lowest robustness, but fastest in execution.
	
	\subsection{SIFT}
		\subsubsection{Introduction}
		\subsubsection{Performance}
			$$\textrm{Add time data}$$
		\subsubsection{Pros \& Cons}
	
	\subsection{SURF}
		\subsubsection{Introduction}
		\subsubsection{Performance}
			$$\textrm{Add time data}$$
		\subsubsection{Pros \& Cons}
	
	\subsection{ORB}
		\subsubsection{Introduction}
		\subsubsection{Performance}
			$$\textrm{Add time data}$$
		\subsubsection{Pros \& Cons}
	
	\subsection{FREAK}
		\subsubsection{Introduction}
		\subsubsection{Performance}
			$$\textrm{Add time data}$$
		\subsubsection{Pros \& Cons}

\section{Comparison between each methods}
\emph{crest a chart here}