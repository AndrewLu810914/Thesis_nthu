\chapter{Comparison Between Other Methods}
\label{c:comparision}

\section{Different Feature Point Extraction \& Point Matching Methods}
	% test functions provides in opencv
	% https://docs.opencv.org/3.3.1/d5/d51/group__features2d__main.html
	In this section, we'll discuss how different feature point performs in keyboard defect detection scenario.
	Since the bottleneck of this project is the feature point detection method, we tried some common methods, and below is the observation \& time comparison between different approach.
	\subsection{Experiment Method}
		\subsubsection{Problems}
			We'll discuss the robustness aganest different kinds of interference this system would encountered in production line.
			The interference most likely happen would be 
			unexcepted rotate \& translate when the operator put keyboard on the fixture, 
			LED failure on the main board of keyboard , 
			lettering defects on the key caps during the production of key caps 
			and noise from camera.
		\subsubsection{Methods}
			We'll simulate the problems just mentioned.\\
			For unexcepted rotate \& translate, \\
			For LED failure, \\
			For lettering defects on the key caps, \\
			For noise from camera, \\



	\subsection{Simple Blob Detector}
		\subsubsection{Introduction}
			The blob feature detector is an simple approach for feature point detection. We choose the centroid of each selected blob as the feature point. Matching feature points with nearest neighborhood. 
			Since we have a fixture that can guarantee the keyboard does not have shift or rotation more then a certain amount, we can assume that the feature point matched is the nearest one.
		\subsubsection{Performance}
			$$\textrm{Add time data}$$
		\subsubsection{Pros \& Cons}
			This method has lowest robustness in experiment, but fastest in execution.

	\subsection{SIFT}
		\subsubsection{Introduction}
		SIFT feature is proposed by Lowe \cite{lowe2004distinctive} et al., having the state of the art robustness in most scenarios \cite{karami2017image}, but slower than any other methods on execution.
		\subsubsection{Performance}
			$$\textrm{Add time data}$$
		\subsubsection{Pros \& Cons}
	
	\subsection{SURF}
		\subsubsection{Introduction}
		SURF is a upgrade version based on SIFT, witch is designed to solve the speed issue of SIFT. 
		And in the scenario of this thesis, this method meets the requirement on the stability without wasting too much time on calculation.
		\subsubsection{Performance}
			$$\textrm{Add time data}$$
		\subsubsection{Pros \& Cons}

	\subsection{ORB}
		\subsubsection{Introduction}
		ORB is a
		\subsubsection{Performance}
			$$\textrm{Add time data}$$
		\subsubsection{Pros \& Cons}

	\subsection{AKAZE}
		\subsubsection{Introduction}
		AKAZE is a
		\subsubsection{Performance}
			$$\textrm{Add time data}$$
		\subsubsection{Pros \& Cons}

\section{Comparison between each methods}
\emph{crest a chart here}