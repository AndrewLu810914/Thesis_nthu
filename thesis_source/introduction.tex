\chapter{Introduction}
\label{c:intro}

% 動機
\section{Motivation}
Since the cost of existing software solution is too high, and has little flexibility while tunning algorithms.
The

\subsection{Previous Solution}
add previous work here.



\section{Related Works}
\label{section:related-work}
add referenced paper here and write some comment

\section{Goal}
\label{section:goal}
Design a system that is both cost efficient and time efficient. Could identify lettering defect and LED light defect on keyboard.

\section{Divide and Conquer}
% \subsection{Definition}
% \label{section:divide-and-conquer}

% Fig.~\ref{fig:DnC} shows the concepts of \emph{divide and conquer} (D\&C). D\&C is an algorithm design paradigm that breaks a complex problem into a couple of relatively simple subproblems, to \emph{divide}, then solves them respectively, to \emph{conquer}. Before conquering, the problem will be divided recursively until it is simple enough to be processed. Finally, the solutions to the subproblems will be merged as those to the original problem.

% \begin{figure}[!htb]
%     \centering
%     \includegraphics[width=\textwidth]{figsrc/DnC.png}
%     \caption{A diagram showing how divide and conquer works.\label{fig:DnC}}
% \end{figure}

% 論文貢獻
\subsection{Main Contribution of This Dissertation}
\label{subsec:advantages}

\subsubsection{Reducing the Difficulty of Problems}

\subsubsection{Independence of Subproblems}

\subsubsection{Parallelism}


% Nowadays, a processor usually has multiple cores, and lots of computational tasks are implemented to be executed with parallel programs. In D\&C algorithm, the functions solving split subproblems are identically designed. With high independence and similar operations between subproblems, it is a good strategy to process them simultaneously. In other word, the original problem is suitable to be solved with \emph{SIMD (Single-Instruction-Multiple-Data)} parallel programs.
